\renewcommand{\abstractname}{Abstract} % Veränderter Name für das Abstract
\begin{abstract}
\begin{addmargin}[1.5cm]{1.5cm}        % Erhöhte Ränder, für Abstract Look
\thispagestyle{plain}                  % Seitenzahl auf der Abstract Seite

\begin{center}
\small\textit{- English -}             % Angabe der Sprache für das Abstract
\end{center}

\vspace{0.25cm}

This work compares several U-Net-based architectures using various pre-training, fine-tuning, and 
augmentation methods to recognize and semantically segment cycle ways in cities. 
To that end, multiple data sets are created and automatically annotated using a custom algorithm 
and OpenStreetMap data. In order to evaluate the results, a buffered form of \textit{\acf{IoU}}, 
called \textit{\acf{BIoU}}, which is the rasterized version of the \textit{Quality} measure, is developed. \\
Results show mediocre success for recognizing cycle ways, especially for cities 
with different infrastructure than that used during training. 
Potentially, better annotations with a greater variety of cities in terms of infrastructure 
could improve the results. Furthermore, semantic segmentation provides more information 
than is required for the task, so alternative approaches, such as directly extracting a graph 
of cycle ways, may yield better results.
The impact of different architectures and training methodologies does not show consistent
performance improvements and is thus deemed insignificant. 

\end{addmargin}
\end{abstract}