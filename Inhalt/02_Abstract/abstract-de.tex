\renewcommand{\abstractname}{Abstract} % Veränderter Name für das Abstract
\begin{abstract}
\begin{addmargin}[1.5cm]{1.5cm}        % Erhöhte Ränder, für Abstract Look
\thispagestyle{plain}                  % Seitenzahl auf der Abstract Seite

\begin{center}
\small\textit{- Deutsch -}             % Angabe der Sprache für das Abstract
\end{center}

\vspace{0.25cm}

In dieser Arbeit wird ein effizienter Algorithmus zur Berechnung der Ähnlichkeit zweier Materialflussgraphen nach der Anzahl gemeinsamer Knotenbezeichner entwickelt. Materialflussgraphen sind \acfp{DAG} mit Knotenbezeichnern, die viele Knoten mit Ausgangsgrad eins besitzen und die einen gemeinsamen Teil-\ac{DAG} haben, wenn sie einen dessen Knotenbezeichner teilen. \\
Diese Entwicklung ist nötig, da das einfache Zählen gemeinsamer Knotenbezeichner sehr ineffizient ist. Stattdessen kann dieser Vorgang deutlich beschleunigt werden, wenn die Eigenschaft des Teil-\acp{DAG} ausgenutzt wird. Hierbei wird zunächst der Graph auf seine Senken und Abzweigungen reduziert (wobei eine Abzweigung ein Knoten mit Ausgangsgrad größer 1 ist) und dann die reduzierten Graphen verglichen. Das ergab eine Beschleunigung um Faktor 10 für das durchschnittliche Problem, wogegen Faktor 83 bis 168 für besonders große Graphen mit besonders wenigen Senken und Abzweigungen erwirkt werden konnte.


\end{addmargin}
\end{abstract}