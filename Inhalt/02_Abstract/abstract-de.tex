\renewcommand{\abstractname}{Abstract} % Veränderter Name für das Abstract
\begin{abstract}
\begin{addmargin}[1.5cm]{1.5cm}        % Erhöhte Ränder, für Abstract Look
\thispagestyle{plain}                  % Seitenzahl auf der Abstract Seite

\begin{center}
\small\textit{- Deutsch -}             % Angabe der Sprache für das Abstract
\end{center}

\vspace{0.25cm}

Diese Arbeit vergleicht einige U-Net-basierte Architekturen und unterschiedliche
Pre-Training-, Fine-Tuning- und Augmentierungs-Methoden zum Erkennen 
und semantisch Segmentieren von städtischen Radwegen. 
Dafür werden mehrere Datensätze erstellt und mittels eines eigens entwickelten Algorithmus 
und OpenStreetMap-Daten automatisch annotiert. Um die Ergebnisse zu bewerten, 
wird eine gepufferte Form der  \textit{\acf{IoU}}, genannt \textit{\acf{BIoU}}, 
welche eine rasterisierte Version des \textit{Quality}-Maßes darstellt, entwickelt. \hspace{2mm} \\
Die Ergebnisse zeigen mittelmäßigen Erfolg beim Erkennen von Fahrradwegen, insbesondere dann, 
wenn Städte mit abweichender Infrastruktur, die nicht während des Trainings verwendet wurden, getestet werden.
Potenziell können bessere Annotationen mit einem Datensatz mit einer größeren Vielfalt an städtischer Infrastruktur 
bessere Ergebnisse erzielen. Außerdem liefert semantische Segmentierung mehr Information 
als für die Aufgabe notwendig ist, weswegen alternative Herangehensweisen, wie das direkte Extrahieren 
eines Graphens der Radwege, bessere Ergebnisse liefern könnten.
Die unterschiedlichen Architekturen und Trainingsmethoden zeigen keine konsequenten 
Verbesserungen der Performanz und werden als nicht signifikant erachtet. 

\end{addmargin}
\end{abstract}