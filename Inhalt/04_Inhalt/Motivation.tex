\chapter{Motivation} \label{sec:motivation}

Die Navigation über Apps wie beispielsweise Google Maps ist inzwischen aus dem Alltag nicht mehr wegzudenken.
Auch für das Zurücklegen einer Strecke mit dem Fahrrad werden entsprechende Routen vorgeschlagen.
Allerdings liegen diese häufig neben großen Straßen und berücksichtigen als vorrangiges Ziel die schnellste Strecke zum Zielpunkt.
Touren müssen eigenständig über das Hinzufügen weiterer Wegpunkte angepasst werden, wenn entspannter ans Ziel gelangt werden möchte.
Um dieses Umplanen zu erleichtern, sollen im Rahmen dieser Studienarbeit untersucht werden, ob es möglich ist 
städtische Radwege über Luftaufnahmen ausfindig zu machen.
Gefundene Radwege sollen dann im Anschluss anhand ihres Umfeldes in verschiedene Kategorien eingruppiert werden, die Rückschlüsse auf die Fahrradfreundlichkeit zulassen.

Die Extraktion von Straßen aus Luftaufnahmen ist bereits
ein gut untersuchtes Teilgebiet der Computer-Vision, wozu es eine Vielzahl an wissenschaftlichen Publikationen gibt
und praxistaugliche Resultate erzielt werden konnten. \\
Allerdings gibt es keinerlei Untersuchungen oder Datensätze zur Erkennung, geschweige denn Extraktion, 
von Radwegen aus Luftbildern. Hierzu ist es nötig, neben dem Erkennen eines befestigten Weges  
diesen auch von Straßen oder Gehwegen abzugrenzen. Selbst für 
Menschen stellt es oft ein schwieriges Unterfangen dar, aufgrund der für das Problem geringen \ac{GSD}, wobei selbst bei modernen 
Aufnahmen ein Radweg nur wenige Pixel breit ist (1 Pixel entspricht 0,20 bis 0,50 Meter), oder ein Radweg von Vegetation wie Bäumen oder von Schatten verdeckt wird.
Diese Arbeit kann also darüber Aufschluss geben, inwiefern eine Abgrenzung zwischen unterschiedlichen Straßentypen oder -bestandteilen mithilfe von Computer Vision
grundsätzlich möglich ist und ob in diesem Anwendungsfall eine Maschine eventuell sogar übermenschliche Performanz 
erreichen kann, um so zum Beispiel auch sehr dunkle, überschattete oder überwachsene Wege zu erkennen. 
Des Weiteren kann diese Arbeit als Anhaltspunkt für verwandte Probleme, wie zum Beispiel das Erkennen von Busspuren 
oder Gehwegen genutzt werden. Auch wird eine Methode vorgeschlagen, wie für solche Fälle ein Datensatz erstellt werden 
kann, mit dem ein Computer-Vision-Modell trainiert werden kann. 


\section{Ziele der Arbeit} \label{mot:ziele}
Das Ziel der Studienarbeit liegt in der Erkennung von Fahrradwegen mithilfe von Computer Vision.
Hierfür werden Satellitenaufnahmen mit einer Bodenauflösung von $20 cm$ verwendet.
Da es für die Erkennung von Fahrradwege keine vorliegenden Datensets gibt, muss dieses selbst erstellt werden.
Die Umsetzung des neuronalen Netzes ist über ein U-Net realisiert.
Im ersten Schritt erfolgt eine reine Klassifikation in Fahrradweg bzw. Nicht-Fahrradweg.
Über das Verändern von Parametern wird versucht, das Netz zu optimieren.
Die Überführung in verschiedene Kategorien soll erst erfolgen, wenn ausreichend zuverlässig Wege identifiziert werden können.

\section{Strukturierung} \label{mot:strukt}

Zu Beginn der Arbeit wird auf den aktuellen Stand der Technik eingegangen.
Der Fokus liegt hierbei auf Sachverhalten, die Bestandteil der Studienarbeit oder wichtig für das Verständnis sind.
So werden zunächst verschiedene Arten von Bilderkennung aufgegriffen.
Im späteren Verlauf verwendete Metriken werden vorgestellt.
Die Struktur und Architektur eines U-Nets wird beschrieben.
Da es bereits Datensätze und Implementierungen zum Segmentieren von Straßen gibt, folgt eine Erläuterung des Transfer-Learning-Ansatzes.
Die Datensätze werden ebenfalls kurz vorgestellt und auf die Unterschiede zwischen ihnen eingegangen.

Bei der Konzeption wird ein Pretraining mithilfe der Straßen-Datensätze durchgeführt.
Im Anschluss erfolgt die Erstellung eines eigenen Datensatzes für Fahrradwege.
Die Architektur des zu implementierenden Netzes wird mit seinen Hyperparametern erläutert.
Zu messende Werte zum Einschätzen der Güte werden festgelegt.
Die Implementierung realisiert die in der Konzeption entworfenen Netze und generiert den Datensatz.

Die Ergebnisse werden anschließend vorgestellt und die einzelnen Varianten miteinander verglichen.
Als Bewertungsbasis werden die definierten Metriken verwendet.
Wichtige Erkenntnisse werden hervorgehoben. Zum Schluss folgt eine kritische Reflexion in Zusammenhang mit einer Zusammenfassung sowie ein Ausblick der Arbeit.