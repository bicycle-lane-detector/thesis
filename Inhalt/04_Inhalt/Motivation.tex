\chapter{Motivation}

Die Navigation über Apps wie beispielsweise Google Maps ist inzwischen aus dem Alltag nicht mehr wegzudenken.
Auch für das Zurücklegen einer Strecke mit dem Fahrrad werden entsprechende Routen vorgeschlagen.
Allerdings liegen diese häufig neben großen Straßen und berücksichtigen als vorrangiges Ziel die schnellste Strecke zum Zielpunkt.
Touren müssen eigenständig über das Hinzufügen weiterer Wegpunkte angepasst werden, wenn man entspannter an sein Ziel gelangen möchte.
Um dieses Umplanen zu erleichtern, sollen im Rahmen dieser Studienarbeit städtische Radwege über Luftaufnahmen ausfindig gemacht werden.
Die Erkennung erfolgt mithilfe von Computer Vision.
Gefundene Radwege sollen dann im Anschluss anhand ihres Umfeldes in verschiedene Kategorien eingruppiert werden, die Rückschlüsse auf die Fahrradfreundlichkeit zulassen.
Die jeweiligen Zustände der einzelnen Radwege bleiben unberücksichtigt.
Ebenso werden Fahrradstraßen und 30er-Zonen ausgenommen, obwohl diese angenehm für Radfahrende sind.
Grund hierfür ist die erschwerte Unterscheidung von anderen Straßen aus Satellitenansicht.

Für die Erkennung von Straßen aus Luftaufnahmen gibt es bereits entsprechende KIs.
Herausforderungen stellen dort eine teilweise oder vollständige Verdeckungen durch Brücken, Bäume etc. dar.
Zusätzlich sorgen Schatten für unterschiedliche Beleuchtungen, welche sich auf das Gesamterscheinungsbild auswirken \cite{Azimi.2018}.
Dazu kommen noch fahrradspezifische Probleme:

\begin{itemize}
	\item Fahrradwege können verschiedenste Untergründe von Asphalt bis Feldweg haben.
			Das Erscheinungsbild ist dementsprechend vielfältig.
	\item Sie existieren in mehreren Varianten: als von der Fahrbahn abgetrennte Fahrspur, geteilter Rad- und Fußweg oder ein eigener Radweg.
			Gerade die Abgrenzung von Fußgängerwegen ist schwierig, da sich diese ebenfalls häufig in Straßennähe befinden und eine ähnliche Breite aufweisen.
			Selbst für Menschen stellt es eine Herausforderung dar, Fahrradwege in den Satellitenaufnahmen zu identifizieren.
	\item Durch ihre geringe Breite im Vergleich zu Straßen, muss die \textit{\ac{GSD}} klein genug gewählt werden, dass sich Fahrradwege überhaupt erkennen lassen.
\end{itemize}

\section{Ziele der Arbeit} \label{mot:ziele}
Das Ziel der Studienarbeit liegt in der Erkennung von Fahrradwegen mithilfe von Computer Vision.
Hierfür werden Satellitenaufnahmen mit einer Bodenauflösung von $20 cm$ verwendet.
Da es für die Erkennung von Fahrradwege keine vorliegenden Datensets gibt, muss dieses selbst erstellt werden.
Die Umsetzung des neuronalen Netzes ist über ein U-Net realisiert.
Im ersten Schritt erfolgt eine reine Klassifikation in Fahrradweg bzw. Nicht-Fahrradweg.
Über das Verändern von Parametern wird versucht, das Netz zu optimieren.
Die Überführung in verschiedene Kategorien soll erst erfolgen, wenn ausreichend zuverlässig Wege identifiziert werden können.

\section{Strukturierung} \label{mot:strukt}

Zu Beginn der Arbeit wird auf den aktuellen Stand der Technik eingegangen.
Der Fokus liegt hierbei auf Sachverhalten, die Bestandteil der Studienarbeit oder wichtig für das Verständnis sind.
So werden zunächst verschiedene Arten von Bilderkennung aufgegriffen.
Im späteren Verlauf verwendete Metriken werden vorgestellt.
Die Struktur und Architektur eines U-Nets wird beschrieben.
Da es bereits Datensätze und Implementierungen zum Segmentieren von Straßen gibt, folgt eine Erläuterung des Transfer-Learning-Ansatzes.
Die Datensätze werden ebenfalls kurz vorgestellt und auf die Unterschiede zwischen ihnen eingegangen.

Bei der Konzeption wird ein Pretraining mithilfe der Straßen-Datensätze durchgeführt.
Im Anschluss erfolgt die Erstellung eines eigenen Datensatzes für Fahrradwege.
Die Architektur des zu implementierenden Netzes wird mit seinen Hyperparametern erläutert.
Zu messende Werte zum Einschätzen der Güte werden festgelegt.
Die Implementierung realisiert die in der Konzeption entworfenen Netze und generiert den Datensatz.

Die Ergebnisse werden anschließend vorgestellt und die einzelnen Varianten miteinander verglichen.
Als Bewertungsbasis werden die definierten Metriken verwendet.
Wichtige Erkenntnisse werden hervorgehoben. Zum Schluss folgt eine kritische Reflexion in Zusammenhang mit einer Zusammenfassung sowie ein Ausblick der Arbeit.