\chapter{Fazit}

Abschließend lässt sich zusammenfassen, dass die Radwegerkennung auf dem BikeSat und Wolfsburg-Datensatz 
ausreichend gut funktioniert hat, aber auf Städten anderer Infrastruktur, wie Karlsruhe, eher mangelhaft.
Die Übertragbarkeit von Ergebnissen zwischen unterschiedlichen Städten und Ländern ist daher, zumindest 
für den Fall der Radwegerkennung, eher schwierig. Mit dem BikeSat-, Wolfsburg- und Karlsruhe-Datensatz 
sind allerdings Datensätze geschaffen, die weiter für die Radwegerkennung aus Luftbildern verwendet 
werden können, um z.B. andere Herangehensweisen zu untersuchen. Die Datensätze sind dabei vom Umfang 
vergleichbar mit Benchmark-Datensätzen der Straßenerkennung. \\
Außerdem zeigt die Arbeit, dass Erkennung von feineren Strukturen der Infrastruktur von Städten, als Straßen 
möglich ist. Für die bessere und intuitivere Einordnung von Ergebnissen der semantischen Segmentierung, 
bei denen die exakte Position weniger relevant ist, zeigt sich die in dieser Arbeit aus der Quality entwickelte \ac{BIoU} 
als nützliches Werkzeug und Bewertungsmaß, welches im praktischen Einsatz die Erwartungen erfüllt. \\
Das überwachte Pre-Training auf Straßendatensätzen und unterschiedliche Modellarchitekturen 
erzeugen keine signifikanten Unterschiede in der Performance. 

Als Zukunftsausblick können auf Basis dieser Arbeit andere Infrastruktur-Elemente in Städten erkannt werden, 
wie Busspuren, Gehwege, Parkplätze am Straßenrand oder sonstiges. 
Möglicherweise kann mit dem direkten Extrahieren der Graphen von z.B. Radwegen einige Probleme mit dem automatischen 
Annotieren der Datensätze umgangen werden und somit bessere Ergebnisse erzielt werden. 
Auch könnte ein unüberwachtes Pre-Training auf einem sehr großen Datensatz mit sehr vielen unterschiedlichen 
Städten einen Vorteil bringen. Außerdem kann untersucht werden, wie viel Einfluss die unterschiedliche 
Infrastruktur hat, und inwiefern diese die Ergebnisse verschlechtert. 
Des Weiteren sollte beachtet werden, dass die Datensätze zur Radwegerkennung in dieser Arbeit 
eine \ac{GSD} von 20 $\frac{cm}{Pixel}$ aufweisen, weswegen auf Zoom-Augmentierung verzichtet wird. 
Sollten unterschiedliche \acp{GSD} in zukünftigen Arbeiten verwendet werden, sollte diese Augmentierung 
untersucht werden. 
